\documentclass[11pt]{article}
\author{Phan-Anh Nguyen\\
		286049} 
\title{Geometry Processing Lab 2012\\
	   Anisotropic Filtering of Non-Linear Surface Features}
\begin{document}
\maketitle



\section{Introduction}

Nowadays, geometric data acquired through imaging or scanning devices has grown rapidly due to advances in technology, making it affordable in many aspects of our lives. However, when dealing with real data we always have to cope with measurement error which brings high frequency noise to our geometric models. Many researches have been conducted in order to remove noise from a scanned model while trying to preserve the underlying sampled surface. One of the seminal results was the work of Taubin et al. \cite{Taubin:1995:SPA:218380.218473} in which they use a signal processing approach to derive the Laplacian operator acting as a low-pass filter on the geometric signal. Even though the Laplacian operator is a powerful tool to remove high frequency noise, it isotropic behaviour makes it unable to preserve sharp features. Hildebrandt and Polthier \cite{Hildebrandt04anisotropicfiltering} have developed an anisotropic method which can preserve high curvature features in a certain direction while suppressing unwanted curvature peaks in the other directions. This method makes it possible to denoise arbitrary surface meshes whereas non-linear geometric features e.g. curved surface regions and feature lines are preserved. This lab report explores and elaborates theory and practice needed to implement the prescribed mean curvature flow proposed by Hildebrandt and Polthier \cite{Hildebrandt04anisotropicfiltering}.

\section{Mean Curvature}



\section{Conclusion}





\bibliographystyle{alpha}	% (uses file "plain.bst")
\bibliography{bibliography}		% expects file "myrefs.bib"
\end{document}