\documentclass[11pt]{article}

\usepackage{graphicx}
\usepackage{hyperref}
\usepackage{amsmath, amsthm, amssymb}

\author{Phan-Anh Nguyen\\
		286049} 
		
\title{Geometry Processing Lab 2012\\
	   Anisotropic Filtering of Non-Linear Surface Features}
	   
\begin{document}
\maketitle



\section{Introduction}

Nowadays, geometric data acquired through imaging or scanning devices has grown rapidly due to advances in technology, making it affordable in many aspects of our lives. However, when dealing with real data we always have to cope with measurement error which brings high frequency noise to our geometric models. Many researches have been conducted in order to remove noise from a scanned model while trying to preserve the underlying sampled surface. One of the seminal results was the work of Taubin et al. \cite{Taubin:1995:SPA:218380.218473} in which they use a signal processing approach to derive the Laplacian operator acting as a low-pass filter on the geometric signal. Even though the Laplacian operator is a powerful tool to remove high frequency noise, it isotropic behaviour makes it unable to preserve sharp features. Hildebrandt and Polthier \cite{Hildebrandt04anisotropicfiltering} have developed an anisotropic method which can preserve high curvature features in a certain direction while suppressing unwanted curvature peaks in the other directions. This method makes it possible to denoise arbitrary surface meshes whereas non-linear geometric features e.g. curved surface regions and feature lines are preserved. This lab report explores and elaborates theory and practice needed to implement the prescribed mean curvature flow proposed by Hildebrandt and Polthier \cite{Hildebrandt04anisotropicfiltering}.

\section{Smoothing Principle}

\begin{figure}[htbp]
  \begin{minipage}[b]{0.45\linewidth}
    \centering
    \includegraphics[width=\textwidth]{umbrella.png}
    \caption{smooth a vertex p by moving it in the direction of the mean curvature vector $\vec{H}(p)$}
    \label{fig:umbrella}
  \end{minipage}
  \hspace{0.5cm}
  \begin{minipage}[b]{0.45\linewidth}
    \centering
    \includegraphics[width=\linewidth]{cotangent.png}
    \caption{Cotangent weights}
    \label{fig:cotangent}
  \end{minipage}
\end{figure}

A very intuitive smoothing operator one can think of is to move a vertex $p$ to the center of gravity (c.o.g) of its one-ring neighbors $N_1(p)$:

\begin{equation}
p \leftarrow \frac{1}{\mid N_1(p) \mid}\sum\limits_{q \in N_1(p)}q = p - \underbrace{\frac{1}{\mid N_1(p) \mid}\sum\limits_{q \in N_1(p)}(p - q)}_{\bigtriangleup p}
\label{eq:update}
\end{equation}

Equation $\ref{eq:update}$ reveals the update form of the smoothing operator in which the old vertex is moved by an amount of the update vector $\bigtriangleup p$ to the new position. The update vector $\bigtriangleup p$ can be generalized to have arbitrary weights over the 1-ring $\sum\limits_{q \in N_1(p)}w_q(p - q)$ other than uniform weights as in the equation $\ref{eq:update}$. In fact, we can choose the weights such that the update vector points in the direction normal to the mesh surface. Such an update formula was first proposed by Pinkall and Polthier \cite{Pinkall93computingdiscrete} known as cotangent weights:

\begin{equation}
\bigtriangledown_p\ area = \vec{H}(p) = 1/2\sum\limits_{q \in N_1(p)}{(cot\alpha_{q} + cot\beta_{q})(p-q)}
\label{eq:cotangent}
\end{equation}

Where $\vec{H}(p)$ is the mean curvature vector at a vertex $p$ and equal the gradient of the area functional $\bigtriangledown_p\ area$ at that vertex. Figure $\ref{fig:umbrella}$ and $\ref{fig:cotangent}$ show how the mean curvature vector is calculated.

\section{Anisotropic Mean Curvature}

The first step towards deriving an anisotropic mean curvature formula is to express the vertex mean curvature vector $\ref{eq:cotangent}$ in terms of an edge based mean curvature vector:
\begin{equation}
  \vec{H}(e) = H_e \vec{N}_e
  \label{eq:edgeMC}
\end{equation}
where $N_e = \frac{N_1 + N_2}{\parallel N_1 + N_2 \parallel}$ is the edge normal vector as shown in figure $\ref{fig:curvature}$ and $H_e = 2\mid e \mid \cos \frac{\theta_e}{2}$ is the edge mean curvature which depends on the dihedral angle $\theta_e$ as illustrated in figure $\ref{fig:dihedral}$. Note that the smaller the dihedral angle, the sharper the edge thus resulting in the higher the mean curvature $H_e$. Therefore, the term $H_e$ can be seen as the measurement of the directional curvature of the surface in the direction orthogonal to the edge.

\begin{figure}[htbp]
  \begin{minipage}[b]{0.45\linewidth}
	\centering
	\includegraphics[width=\textwidth]{curvature.png}
	\caption{Edge normal vector $N_e$}
	\label{fig:curvature}
  \end{minipage}
  \hspace{0.5cm}
  \begin{minipage}[b]{0.45\linewidth}
    \centering
    \includegraphics[width=\linewidth]{dihedral.png}
    \caption{Dihedral angle $\theta_e$}
    \label{fig:dihedral}
  \end{minipage}
\end{figure}

It can be shown, in the work of Polthier \cite{PolthierHabilitation}, that the vertex mean curvature vector $\vec{H}(p)$ and the edge mean curvature vector $\vec{H}(e)$ are related by the equation:

\begin{equation}
\vec{H}(p) = \frac{1}{2}\sum\limits_{e = (p, q), q \in N_1(p)}\vec{H}(e)
\end{equation}

The anisotropic mean curvature vector $\vec{H}_A$ at a vertex $p$ is then defined as a weighted sum over the contributions $H_e\vec{N}_e$ at the edges incident to a vertex $p$:

\begin{equation}
\vec{H_A}(p) = 1/2\sum\limits_{e = (p, q), q \in N_1(p)}{w(H_e)H_e \vec{N}(e)}
\label{eq:aniso}
\end{equation}

The weight function $w$ is used to put less weight on feature vertices in order to avoid smoothing sharp features:

\begin{equation}
w_{\lambda, r}(a) = 
\begin{cases}
1 & \text{for } |a| \leq \lambda \\
\dfrac{\lambda^2}{r(\lambda - |a|)^2+\lambda^2} & \text{for } |a| > \lambda
\end{cases}
\end{equation}

The threshold $\lambda$ is used to detect features and the radius $r$ controls the width of the transition between those areas that are smoothed and those that are kept as features. In our implementation, we choose $\lambda = 2 \lambda' \max |e| $ where $\lambda' \in \left[ 0, 1 \right] $ to cover the whole range of the edge mean curvature $H_e$. Following Hildebrandt and Polthier suggestion \cite{Hildebrandt04anisotropicfiltering}, we fix the radius $r$ to 10 to ensure that $w_{\lambda, 10}(2\lambda) < 0.1$.

The smoothing operation is carried out by integrating the flow of the anisotropic mean curvature vector $\vec{H}_A$ with some integration scheme. We first use the explicit Euler method because it is simpler to implement. Discussion about the semi-implicit integration scheme will be presented later on.

Given the mesh $M_h$ with its vertices $\mathcal{P} = \left\lbrace p_1, ..., p_m \right\rbrace $, an explicit iteration step of the anisotropic mean curvature flow is computed as follows:

\begin{equation}
\mathcal{P}^{j+1} = \mathcal{P}^j -sM^{-1}\vec{H_A}(\mathcal{P}^j)
\end{equation}

where $s \in [0, 1]$ is a damping factor controlling the magnitude of the update vector thereby stabilizing the flow. $M^{-1}$ is the inverse of the mass matrix $M \in \mathcal{R}^{m \times m}$ of the mesh $M^j_h$ at time step $j$. It is used to convert the integrated mean curvature vector into a piecewise linear vector field:

\begin{equation}
M_{pq} = 
\begin{cases} \dfrac{1}{6}area(star\ p), & \mbox{if } p=q \\ 
\dfrac{1}{12}area(star\ e), & \mbox{if there is an edge } e=(p, q) \\
0, & \mbox{otherwise} \end{cases}
\label{eq:mass_matrix}
\end{equation}

If we use a diagonalization of $M$ called the lumped mass matrix with diagonal elements $M_{pp} = \frac{1}{3} area(star\ p)$, then the integration step for each vertex $p$ is derived as follows:

\begin{equation}
p^{j+1} = p^{j} - \dfrac{3s}{area(star\ p^j)}\vec{H_A}(p^j)
\end{equation}

One disadvantage of the lumped mass matrix is that it gives an unstable solution. To overcome this problem we have to use a very small time step leading to a slow convergence rate. In contrast, the full mass matrix gives a more stable solution, thereby giving a faster convergence rate with the cost of having to compute the inverted mass matrix.

\section{Prescribed Mean Curvature}

Since the mean curvature vector at a vertex is equivalent to its area gradient, applying the mean curvature flow on the mesh has the same effect as minimizing its surface area. This leads to the problem of surface shrinkage. This issue also occurs in the anisotropic case. Particularly, the anisotropic smoothing slows down the smoothing process in regions with high curvature, causing
deformations of the surface as shown in figure $\ref{fig:amc_pmc}$.

\begin{figure}[htb]
\centering
\includegraphics[width=\textwidth]{amc_pmc.png}
\caption{The Anisotropic Mean Curvature flow (left) contracts the interior of the bunny's ears making it to deform. In contrast, the Prescribed Mean Curvature flow (right) converges to a stable surface.}
\label{fig:amc_pmc}
\end{figure}

To circumvent this problem, Hildebrandt and Polthier \cite{Hildebrandt04anisotropicfiltering} used the so-called prescribed mean curvature flow (PMC) to evolve the surface towards a surface having a prescribed mean curvature. This process includes two steps. In the first step, the surface mean curvature is computed and then this scalar field is smoothed. In the second step, the PMC flow is applied to let the surface evolve towards a surface with this precomputed mean curvature. 

Explain volume gradient in detail.

\section{Implementation and Results}

use OpenFlipper, Eigen library, Cholmod suite sparse to invert mass matrix

\section{Implicit Integration of the Flow}

Matrix form: first attempt Ha, second attempt Taylor approximation.

\section{Conclusion and Future Work}

summary the main idea, pose problem to be solved in the future

\bibliographystyle{alpha}	% (uses file "plain.bst")
\bibliography{bibliography}		% expects file "myrefs.bib"
\end{document}