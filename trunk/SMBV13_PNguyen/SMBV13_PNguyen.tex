%%%%%%%%%%%%%%%%%%%%%%%%%%%%%%%%%%%%%%%%%%%%%%%%%%%%%%%%%%%%
% Dokument-Einstellungen
\documentclass{SMBV13}

%%%%%%%%%%%%%%%%%%%%%%%%%%%%%%%%%%%%%%%%%%%%%%%%%%%%%%%%%%%%
%-----------------------------------------------------------
% Hier beginnt das eigentliche Dokument
\begin{document}

\title{Richtlinien zur Erstellung der druckfertigen Beitr�ge f�r das Hauptseminar Medizinische Bildverarbeitung im Sommersemester 2013}

\author{Elke Musterfrau}

\maketitle

%%%%%%%%%%%%%%%%%%%%%%%%%%%%%%%%%%%%%%%%%%%%%%%%%%%%%%%%%%%%
%-----------------------------------------------------------% Zusammenfassung

\begin{abstract}%
Zum Seminar Medizinische Bildverarbeitung ist die Herausgabe eines Seminarbandes geplant, der alle Ausarbeitungen umfasst. Um die Lesbarkeit zu erh�hen und ein einheitliches Layout der Ausarbeitungen zu gew�hrleisten, bitten wir die Seminaristen, die hier zusammengefassten Vorschriften zu beachten und einzuhalten. Nur so k�nnen die Druckkosten und der Aufwand der Weiterverarbeitung gering gehalten werden.
\end{abstract}

\keywords{Proceedings, Druckformate, Seitenlayout, Fristen}


%%%%%%%%%%%%%%%%%%%%%%%%%%%%%%%%%%%%%%%%%%%%%%%%%%%%%%%%%%%%
%-----------------------------------------------------------
%
\section{Einleitung}

In der Hausdruckerei des Klinikums wird ein Seminarband mit allen
Ausarbeitungen im DIN-A4 Format gedruckt. Jeder Seminarteilnehmer wird �ber
die Fertigstellung informiert und kann dann einen solchen Band im Sekretariat des Instituts abholen.



%%%%%%%%%%%%%%%%%%%%%%%%%%%%%%%%%%%%%%%%%%%%%%%%%%%%%%%%%%%%
%-----------------------------------------------------------
%
\section{Generelles Format und Seitenlayout}


%%%%%%%%%%%%%%%%%%%%%%%%%%%%%%%%%%%%%%%%%%%%%%%%%%%%%%%%%%%%
\subsection{Seitenrand}

Bitte erstellen Sie Ihre Beitr�ge in \LaTeX\ mit der Dokumtenklasse SMBV13 (kopieren Sie die Datei "'SMBV13.cls"' in Ihr Arbeitsverzeichniss und laden Sie die Dokumentenklasse wie in diesem Beispieldokument). Die Klasse sorgt f�r die richtigen Seitenr�nder, sowie die korrekten Schriftstyle. Bitte verwenden Sie die Sourcen dieses Dokuments als Vorlage f�r Ihre Ausarbeitung.


%%%%%%%%%%%%%%%%%%%%%%%%%%%%%%%%%%%%%%%%%%%%%%%%%%%%%%%%%%%%
\subsection{Keine Kopf-- und Fu�zeilen und keine Seitenzahlen}

Die Seitenzahlen im Seminarband werden nicht von den Autoren, sondern vom Institut f�r Medizinische Informatik eingef�gt. Bitte erstellen Sie weder eine Kopf- noch eine Fu�zeile.


%%%%%%%%%%%%%%%%%%%%%%%%%%%%%%%%%%%%%%%%%%%%%%%%%%%%%%%%%%%%
\subsection{Gliederung und Umfang des Beitrages}

Jeder Beitrag muss wie folgt gegliedert sein: 
\begin{itemize}
\item
Deckblatt mit dem Thema des Seminares, Vorname und Name des Seminaristen und
Inhaltsverzeichnis (s. Blatt 1 dieser Richtlinien)
\item
neue Seite, beginnend mit Zusammenfassung (maximal 100 Worte), 3-5 Keywords, Text und
Literatur.
\end{itemize}


%%%%%%%%%%%%%%%%%%%%%%%%%%%%%%%%%%%%%%%%%%%%%%%%%%%%%%%%%%%%
\subsection{Schriftarten}

Die Klassendefinition sorgt f�r die Verwendung der richtigen Schriftarten und Gr��en.

%%%%%%%%%%%%%%%%%%%%%%%%%%%%%%%%%%%%%%%%%%%%%%%%%%%%%%%%%%%%
\subsection{Abs�tze}

Neue Abs�tze werden am Anfang um 5mm einger�ckt. Der Zeilenabstand
zwischen den Abs�tzen wird nicht ge�ndert. Bitte f�gen Sie keine 
Leerzeilen zwischen einzelnen Abs�tzen ein.

Um Stellen hervorzuheben, werden diese 
{\em kursiv} gesetzt. Bitte vermeiden Sie Unterstreichungen oder Fettdruck.

%%%%%%%%%%%%%%%%%%%%%%%%%%%%%%%%%%%%%%%%%%%%%%%%%%%%%%%%%%%%
\subsection{Abbildungen und Tabellen}

Abbildungen werden elektronisch in das Dokument integriert und k�nnen im Text
beliebig platziert werden.  Jede Abbildung oder Tabelle muss nummeriert sein und eine Unterschrift erhalten (Tabellen eine �berschrift). Im Text muss auf die Abbildung bzw. Tabelle explizit verwiesen
werden.

Beachten Sie, dass sich gerasterte Grauwertvorlagen durch die
Kopie in der Druckerei in der Qualit�t verschlechtern. Aus demselben
Grund sollte man in Diagrammen auf Schattierungen des
Hintergrundes verzichten und stattdessen Schraffuren einsetzen.
Alle Grafiken m�ssen bei der Einreichung unabh�ngig vom ansonsten verwendeten Format auch im {\em eps}-Format vorliegen.

Im Text muss auf Abbildungen und Tabellen explizit verwiesen werden. Auch die Aussage, die mit dem Objekt visualisiert werden soll, muss im Text explizit genannt werden. S�tze wie: "`Abb. 1 zeigt das Ergebnis"'
alleine sind nicht ausreichend.
Um mehrere Abbildungen unter einer Bildunterschrift nebeneinander zu setzen, kann das Packet {\em subfigure} verwendet werde (\figurename~\ref{SMBV13_EMuster_fig01}).

\begin{figure}[b]
 	\centering
		\subfigure[Eins]{\includegraphics[width=0.3\textwidth]{Bilder/ScreenShot005}}
		\subfigure[Zwei]{\includegraphics[width=0.3\textwidth]{Bilder/ScreenShot005}}
		\subfigure[Drei]{\includegraphics[width=0.3\textwidth]{Bilder/ScreenShot005}}
 		\caption{\label{SMBV13_EMuster_fig01}Beispiel f�r die Einbindung mehrerer Graphiken unter eine Unterschrift.}
\end{figure}

%%%%%%%%%%%%%%%%%%%%%%%%%%%%%%%%%%%%%%%%%%%%%%%%%%%%%%%%%%%%
\subsection{Nummerierungen und Verweise}

Bitte verwenden Sie ausschlie�lich dezimale Einteilungen f�r die
Nummerierung der �berschriften, Abbildungen, Tabellen, Gleichungen, 
Literaturverweise und sonstigen Elementen. 

Die Nummerierung darf dabei nicht von Hand, sondern mit den von 
\LaTeX\ zur Verf�gung gestellten Automatismen erfolgen. 
Nur so ist ein konsistente Nummerierung �ber den gesamten 
Seminarband zu erreichen. Au�erdem erspart eine automatische 
Nummerierung Arbeit, wenn w�hrend der Beitragserstellung Elemente eingef�gt, 
entfernt oder umsortiert werden.

Die Verweise auf nummerierte Elemente sind aus denselben Gr�nden ebenfalls 
automatisch zu generieren.


%%%%%%%%%%%%%%%%%%%%%%%%%%%%%%%%%%%%%%%%%%%%%%%%%%%%%%%%%%%%
\subsection{Formeln}

Mathematische Formeln werden innerhalb der Seite zentriert ausgerichtet
\begin{equation}
c=sqrt(a^2+b^2)
\end{equation}
und am rechten Rand auf der H�he des Gleichheitszeichens fortlaufend nummeriert.
Einfachere Formeln, wie  $x+y=z$, k�nnen auch fortlaufend im Text
erscheinen.


%%%%%%%%%%%%%%%%%%%%%%%%%%%%%%%%%%%%%%%%%%%%%%%%%%%%%%%%%%%%
\subsection{Textanmerkungen / Fu�noten}

Bitte verzichten Sie aus Gr�nden der Lesbarkeit auf Fu�noten und sonstige
Textanmerkungen (Diese k�nnen in Klammern direkt in den Text eingef�gt
werden).


%%%%%%%%%%%%%%%%%%%%%%%%%%%%%%%%%%%%%%%%%%%%%%%%%%%%%%%%%%%%
\subsection{Literaturangaben}


\cite{mortensen1995intelligent} \cite{fripp2007automatic}
\cite{chan2001active} \cite{nowozin2009global}
\cite{gleich2006hierarchical}


%%%%%%%%%%%%%%%%%%%%%%%%%%%%%%%%%%%%%%%%%%%%%%%%%%%%%%%%%%%%
%-----------------------------------------------------------
%
\def\refname{Literatur}
\begin{thebibliography}{AA}
                 
\bibitem{mortensen1995intelligent} Mortensen, Eric N., and William A. Barrett: Intelligent scissors for image composition. Proceedings of the 22nd annual conference on Computer graphics and interactive techniques. ACM, 1995.

\bibitem{fripp2007automatic} Fripp, Jurgen and Crozier, Stuart and Warfield, Simon and Ourselin, S{\'e}bastien: Automatic segmentation of articular cartilage in magnetic resonance images of the knee. Medical Image Computing and Computer-Assisted Intervention--MICCAI 2007 (2007): 186--194.

\bibitem{chan2001active} Chan, Tony F and Vese, Luminita A: Active contours without edges. Image Processing, IEEE Transactions on 10.2 (2001): 266--277.

\bibitem{nowozin2009global} Nowozin, Sebastian, and Christoph H. Lampert: Global connectivity potentials for random field models. Computer Vision and Pattern Recognition, 2009. CVPR 2009. IEEE Conference on. IEEE, 2009.

\bibitem{gleich2006hierarchical} Gleich, David: Hierarchical directed spectral graph partitioning. (2006).

\end{thebibliography}

%\noindent
%\begin{picture}(160,242)
%\put(0,0){\framebox(160,242){}}
%\end{picture}

\end{document}
