%%%%%%%%%%%%%%%%%%%%%%%%%%%%%%%%%%%%%%%%%%%%%%%%%%%%%%%%%%%%
% Dokument-Einstellungen
\documentclass{SMBV13}

%%%%%%%%%%%%%%%%%%%%%%%%%%%%%%%%%%%%%%%%%%%%%%%%%%%%%%%%%%%%
%-----------------------------------------------------------
% Hier beginnt das eigentliche Dokument
\begin{document}

\title{Richtlinien zur Erstellung der druckfertigen Beitr�ge f�r das Hauptseminar Medizinische Bildverarbeitung im Sommersemester 2013}

\author{Elke Musterfrau}

\maketitle

%%%%%%%%%%%%%%%%%%%%%%%%%%%%%%%%%%%%%%%%%%%%%%%%%%%%%%%%%%%%
%-----------------------------------------------------------% Zusammenfassung

\begin{abstract}%
Zum Seminar Medizinische Bildverarbeitung ist die Herausgabe eines Seminarbandes geplant, der alle Ausarbeitungen umfasst. Um die Lesbarkeit zu erh�hen und ein einheitliches Layout der Ausarbeitungen zu gew�hrleisten, bitten wir die Seminaristen, die hier zusammengefassten Vorschriften zu beachten und einzuhalten. Nur so k�nnen die Druckkosten und der Aufwand der Weiterverarbeitung gering gehalten werden.
\end{abstract}

\keywords{Proceedings, Druckformate, Seitenlayout, Fristen}


%%%%%%%%%%%%%%%%%%%%%%%%%%%%%%%%%%%%%%%%%%%%%%%%%%%%%%%%%%%%
%-----------------------------------------------------------
%
\section{Einleitung}

In der Hausdruckerei des Klinikums wird ein Seminarband mit allen
Ausarbeitungen im DIN-A4 Format gedruckt. Jeder Seminarteilnehmer wird �ber
die Fertigstellung informiert und kann dann einen solchen Band im Sekretariat des Instituts abholen.



%%%%%%%%%%%%%%%%%%%%%%%%%%%%%%%%%%%%%%%%%%%%%%%%%%%%%%%%%%%%
%-----------------------------------------------------------
%
\section{Generelles Format und Seitenlayout}


%%%%%%%%%%%%%%%%%%%%%%%%%%%%%%%%%%%%%%%%%%%%%%%%%%%%%%%%%%%%
\subsection{Seitenrand}

Bitte erstellen Sie Ihre Beitr�ge in \LaTeX\ mit der Dokumtenklasse SMBV13 (kopieren Sie die Datei "'SMBV13.cls"' in Ihr Arbeitsverzeichniss und laden Sie die Dokumentenklasse wie in diesem Beispieldokument). Die Klasse sorgt f�r die richtigen Seitenr�nder, sowie die korrekten Schriftstyle. Bitte verwenden Sie die Sourcen dieses Dokuments als Vorlage f�r Ihre Ausarbeitung.


%%%%%%%%%%%%%%%%%%%%%%%%%%%%%%%%%%%%%%%%%%%%%%%%%%%%%%%%%%%%
\subsection{Keine Kopf-- und Fu�zeilen und keine Seitenzahlen}

Die Seitenzahlen im Seminarband werden nicht von den Autoren, sondern vom Institut f�r Medizinische Informatik eingef�gt. Bitte erstellen Sie weder eine Kopf- noch eine Fu�zeile.


%%%%%%%%%%%%%%%%%%%%%%%%%%%%%%%%%%%%%%%%%%%%%%%%%%%%%%%%%%%%
\subsection{Gliederung und Umfang des Beitrages}

Jeder Beitrag muss wie folgt gegliedert sein: 
\begin{itemize}
\item
Deckblatt mit dem Thema des Seminares, Vorname und Name des Seminaristen und
Inhaltsverzeichnis (s. Blatt 1 dieser Richtlinien)
\item
neue Seite, beginnend mit Zusammenfassung (maximal 100 Worte), 3-5 Keywords, Text und
Literatur.
\end{itemize}


%%%%%%%%%%%%%%%%%%%%%%%%%%%%%%%%%%%%%%%%%%%%%%%%%%%%%%%%%%%%
\subsection{Schriftarten}

Die Klassendefinition sorgt f�r die Verwendung der richtigen Schriftarten und Gr��en.

%%%%%%%%%%%%%%%%%%%%%%%%%%%%%%%%%%%%%%%%%%%%%%%%%%%%%%%%%%%%
\subsection{Abs�tze}

Neue Abs�tze werden am Anfang um 5mm einger�ckt. Der Zeilenabstand
zwischen den Abs�tzen wird nicht ge�ndert. Bitte f�gen Sie keine 
Leerzeilen zwischen einzelnen Abs�tzen ein.

Um Stellen hervorzuheben, werden diese 
{\em kursiv} gesetzt. Bitte vermeiden Sie Unterstreichungen oder Fettdruck.

%%%%%%%%%%%%%%%%%%%%%%%%%%%%%%%%%%%%%%%%%%%%%%%%%%%%%%%%%%%%
\subsection{Abbildungen und Tabellen}

Abbildungen werden elektronisch in das Dokument integriert und k�nnen im Text
beliebig platziert werden.  Jede Abbildung oder Tabelle muss nummeriert sein und eine Unterschrift erhalten (Tabellen eine �berschrift). Im Text muss auf die Abbildung bzw. Tabelle explizit verwiesen
werden.

Beachten Sie, dass sich gerasterte Grauwertvorlagen durch die
Kopie in der Druckerei in der Qualit�t verschlechtern. Aus demselben
Grund sollte man in Diagrammen auf Schattierungen des
Hintergrundes verzichten und stattdessen Schraffuren einsetzen.
Alle Grafiken m�ssen bei der Einreichung unabh�ngig vom ansonsten verwendeten Format auch im {\em eps}-Format vorliegen.

Im Text muss auf Abbildungen und Tabellen explizit verwiesen werden. Auch die Aussage, die mit dem Objekt visualisiert werden soll, muss im Text explizit genannt werden. S�tze wie: "`Abb. 1 zeigt das Ergebnis"'
alleine sind nicht ausreichend.
Um mehrere Abbildungen unter einer Bildunterschrift nebeneinander zu setzen, kann das Packet {\em subfigure} verwendet werde (\figurename~\ref{SMBV13_EMuster_fig01}).

\begin{figure}[b]
 	\centering
		\subfigure[Eins]{\includegraphics[width=0.3\textwidth]{Bilder/ScreenShot005}}
		\subfigure[Zwei]{\includegraphics[width=0.3\textwidth]{Bilder/ScreenShot005}}
		\subfigure[Drei]{\includegraphics[width=0.3\textwidth]{Bilder/ScreenShot005}}
 		\caption{\label{SMBV13_EMuster_fig01}Beispiel f�r die Einbindung mehrerer Graphiken unter eine Unterschrift.}
\end{figure}

%%%%%%%%%%%%%%%%%%%%%%%%%%%%%%%%%%%%%%%%%%%%%%%%%%%%%%%%%%%%
\subsection{Nummerierungen und Verweise}

Bitte verwenden Sie ausschlie�lich dezimale Einteilungen f�r die
Nummerierung der �berschriften, Abbildungen, Tabellen, Gleichungen, 
Literaturverweise und sonstigen Elementen. 

Die Nummerierung darf dabei nicht von Hand, sondern mit den von 
\LaTeX\ zur Verf�gung gestellten Automatismen erfolgen. 
Nur so ist ein konsistente Nummerierung �ber den gesamten 
Seminarband zu erreichen. Au�erdem erspart eine automatische 
Nummerierung Arbeit, wenn w�hrend der Beitragserstellung Elemente eingef�gt, 
entfernt oder umsortiert werden.

Die Verweise auf nummerierte Elemente sind aus denselben Gr�nden ebenfalls 
automatisch zu generieren.


%%%%%%%%%%%%%%%%%%%%%%%%%%%%%%%%%%%%%%%%%%%%%%%%%%%%%%%%%%%%
\subsection{Formeln}

Mathematische Formeln werden innerhalb der Seite zentriert ausgerichtet
\begin{equation}
c=sqrt(a^2+b^2)
\end{equation}
und am rechten Rand auf der H�he des Gleichheitszeichens fortlaufend nummeriert.
Einfachere Formeln, wie  $x+y=z$, k�nnen auch fortlaufend im Text
erscheinen.


%%%%%%%%%%%%%%%%%%%%%%%%%%%%%%%%%%%%%%%%%%%%%%%%%%%%%%%%%%%%
\subsection{Textanmerkungen / Fu�noten}

Bitte verzichten Sie aus Gr�nden der Lesbarkeit auf Fu�noten und sonstige
Textanmerkungen (Diese k�nnen in Klammern direkt in den Text eingef�gt
werden).


%%%%%%%%%%%%%%%%%%%%%%%%%%%%%%%%%%%%%%%%%%%%%%%%%%%%%%%%%%%%
\subsection{Literaturangaben}


\cite{mortensen1995intelligent} \cite{fripp2007automatic}
\cite{chan2001active} \cite{nowozin2009global}
\cite{gleich2006hierarchical} \cite{lempitsky2010fusion}
\cite{kolmogorov2007minimizing} \cite{boykov2004experimental}
\cite{mitchell2002branch} \cite{VijayGrauman2011}
\cite{bishop2006pattern} \cite{shimizu2011automated}
\cite{KenGalShi2011} \cite{nakagomimulti}
\cite{leventon2000statistical} \cite{tsochantaridis2006large}
\cite{maire2008using} \cite{gu2009recognition}
\cite{bay2006surf} \cite{ljubic2006algorithmic}
\cite{shi2000normalized} \cite{zhu2007untangling}
\cite{martin2004learning} \cite{arbelaez2009contours}
\cite{malik2001contour} \cite{leung1998contour}
\cite{cortes1995support} \cite{burges1998tutorial}
\cite{lampert2008beyond}


%%%%%%%%%%%%%%%%%%%%%%%%%%%%%%%%%%%%%%%%%%%%%%%%%%%%%%%%%%%%
%-----------------------------------------------------------
%
\def\refname{Literatur}
\begin{thebibliography}{AA}
                 
\bibitem{mortensen1995intelligent} Mortensen, Eric N., and William A. Barrett: Intelligent scissors for image composition. Proceedings of the 22nd annual conference on Computer graphics and interactive techniques. ACM, 1995.

\bibitem{fripp2007automatic} Fripp, Jurgen and Crozier, Stuart and Warfield, Simon and Ourselin, S{\'e}bastien: Automatic segmentation of articular cartilage in magnetic resonance images of the knee. Medical Image Computing and Computer-Assisted Intervention--MICCAI 2007 (2007): 186--194.

\bibitem{chan2001active} Chan, Tony F and Vese, Luminita A: Active contours without edges. Image Processing, IEEE Transactions on 10.2 (2001): 266--277.

\bibitem{nowozin2009global} Nowozin, Sebastian, and Christoph H. Lampert: Global connectivity potentials for random field models. Computer Vision and Pattern Recognition, 2009. CVPR 2009. IEEE Conference on. IEEE, 2009.

\bibitem{gleich2006hierarchical} Gleich, David: Hierarchical directed spectral graph partitioning. (2006).

\bibitem{lempitsky2010fusion} Lempitsky, Victor, Carsten Rother, Stefan Roth, and Andrew Blake: Fusion moves for markov random field optimization. Pattern Analysis and Machine Intelligence, IEEE Transactions on 32, no. 8 (2010): 1392--1405.

\bibitem{kolmogorov2007minimizing} Kolmogorov, Vladimir, and Carsten Rother: Minimizing nonsubmodular functions with graph cuts-a review. Pattern Analysis and Machine Intelligence, IEEE Transactions on 29.7 (2007): 1274--1279.

\bibitem{boykov2004experimental} Boykov, Yuri, and Vladimir Kolmogorov: An experimental comparison of min-cut/max-flow algorithms for energy minimization in vision. Pattern Analysis and Machine Intelligence, IEEE Transactions on 26.9 (2004): 1124--1137.

\bibitem{mitchell2002branch} Mitchell, John E: Branch-and-cut algorithms for combinatorial optimization problems. Handbook of Applied Optimization (2002): 65--77.

\bibitem{VijayGrauman2011} Vijayanarasimhan, Sudheendra, and Kristen Grauman: Efficient region search for object detection. Computer Vision and Pattern Recognition (CVPR), 2011 IEEE Conference on. IEEE, 2011.

\bibitem{bishop2006pattern} Bishop, Christopher M.: Pattern recognition and machine learning. Vol. 4. No. 4. New York: springer, 2006.

\bibitem{shimizu2011automated} Shimizu, Akinobu, Keita Nakagomi, Takuya Narihira, Hidefumi Kobatake, Shigeru Nawano, Kenji Shinozaki, Koich Ishizu, and Kaori Togashi: Automated segmentation of 3D CT images based on statistical atlas and graph cuts. Medical Computer Vision. Recognition Techniques and Applications in Medical Imaging (2011): 214--223.

\bibitem{KenGalShi2011} Kennedy, Ryan, Jean Gallier, and Jianbo Shi: Contour cut: identifying salient contours in images by solving a Hermitian eigenvalue problem. In Computer Vision and Pattern Recognition (CVPR), 2011 IEEE Conference on, pp. 2065--2072. IEEE, 2011.

\bibitem{nakagomimulti} Nakagomi, Keita, Akinobu Shimizu, Hidefumi Kobatake, Masahiro Yakami, Koji Fujimoto, and Kaori Togashi: Multi-shape graph-cuts and its application to lung segmentation from a chest CT volume.

\bibitem{leventon2000statistical} Leventon, Michael E., W. Eric L. Grimson, and Olivier Faugeras: Statistical shape influence in geodesic active contours. In Computer Vision and Pattern Recognition, 2000. Proceedings. IEEE Conference on, vol. 1, pp. 316--323. IEEE, 2000.

\bibitem{tsochantaridis2006large} Tsochantaridis, Ioannis, Thorsten Joachims, Thomas Hofmann, and Yasemin Altun: Large margin methods for structured and interdependent output variables. Journal of Machine Learning Research 6, no. 2 (2006): 1453.

\bibitem{maire2008using} Maire, M. and Arbel{\'a}ez, P. and Fowlkes, C. and Malik, J.: Using contours to detect and localize junctions in natural images. In Computer Vision and Pattern Recognition, 2008. CVPR 2008. IEEE Conference on, pp. 1--8. IEEE, 2008.

\bibitem{gu2009recognition} Gu, C. and Lim, J.J. and Arbel{\'a}ez, P. and Malik, J.: Recognition using regions. In Computer Vision and Pattern Recognition, 2009. CVPR 2009. IEEE Conference on, pp. 1030--1037. IEEE, 2009.

\bibitem{bay2006surf} Bay, Herbert, Tinne Tuytelaars, and Luc Van Gool: Surf: Speeded up robust features. Computer Vision--ECCV 2006 (2006): 404--417.

\bibitem{ljubic2006algorithmic} Ljubi{\'c}, I. and Weiskircher, R. and Pferschy, U. and Klau, G.W. and Mutzel, P. and Fischetti, M.: An algorithmic framework for the exact solution of the prize--collecting Steiner tree problem. Mathematical Programming 105, no. 2 (2006): 427--449.

\bibitem{shi2000normalized} Shi, Jianbo, and Jitendra Malik: Normalized cuts and image segmentation. Pattern Analysis and Machine Intelligence, IEEE Transactions on 22, no. 8 (2000): 888--905.

\bibitem{zhu2007untangling} Zhu, Qihui, Gang Song, and Jianbo Shi: Untangling cycles for contour grouping. In Computer Vision, 2007. ICCV 2007. IEEE 11th International Conference on, pp. 1--8. IEEE, 2007.

\bibitem{martin2004learning} Martin, David R., Charless C. Fowlkes, and Jitendra Malik: Learning to detect natural image boundaries using local brightness, color, and texture cues. Pattern Analysis and Machine Intelligence, IEEE Transactions on 26, no. 5 (2004): 530--549.

\bibitem{arbelaez2009contours} Arbel{\'a}ez, P. and Maire, M. and Fowlkes, C. and Malik, J.: From contours to regions: An empirical evaluation. In Computer Vision and Pattern Recognition, 2009. CVPR 2009. IEEE Conference on, pp. 2294--2301. IEEE, 2009.

\bibitem{malik2001contour} Malik, Jitendra, Serge Belongie, Thomas Leung, and Jianbo Shi: Contour and texture analysis for image segmentation. International Journal of Computer Vision 43, no. 1 (2001): 7--27.

\bibitem{leung1998contour} Leung, Thomas, and Jitendra Malik: Contour continuity in region based image segmentation. Computer Vision--ECCV'98 (1998): 544--559.

\bibitem{cortes1995support} Cortes, Corinna, and Vladimir Vapnik: Support--vector networks. Machine learning 20, no. 3 (1995): 273--297.

\bibitem{burges1998tutorial} Burges, Christopher JC: A tutorial on support vector machines for pattern recognition. Data mining and knowledge discovery 2, no. 2 (1998): 121--167.

\bibitem{lampert2008beyond} Lampert, Christoph H., Matthew B. Blaschko, and Thomas Hofmann: Beyond sliding windows: Object localization by efficient subwindow search. In Computer Vision and Pattern Recognition, 2008. CVPR 2008. IEEE Conference on, pp. 1--8. IEEE, 2008.

\end{thebibliography}

%\noindent
%\begin{picture}(160,242)
%\put(0,0){\framebox(160,242){}}
%\end{picture}

\end{document}
